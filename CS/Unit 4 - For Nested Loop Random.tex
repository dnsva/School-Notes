

\documentclass{article}

\begin{document}

\section{Random Numbers}

\subsection{Syntax (3 parts)}

\begin{verbatim}
#include <ctime>
#include <cstdlib>
srand(time(0));
\end{verbatim}

\subsection{Using the rand() function}
Between two numbers - [min:max]

\begin{verbatim}
int randNum = rand()%(max-min + 1) + min;
\end{verbatim}

\section{For Loops}

\subsection{Format}
\begin{verbatim}
for(initialization; test; update){ //header
     statement(s) //body
}
\end{verbatim}

\section{Functions}

Built in functions:
\begin{verbatim}
pow(), fabs(), floor(), ceil(), sqrt(), ...
\end{verbatim}

\noindent
Modular Programming - porting from program to program\\

\subsection{The 3 Parts}
1) Function Declaration - (Prototype) - Identifies function name and requirements (before main())
\begin{verbatim}
     int add(int a, int b);
\end{verbatim}
2) Function Definition - Code for the specific task (in this course, definitions should be after main())
\begin{verbatim}
     int add(int a, int b){
          return a + b;
     }
\end{verbatim}
3) Function Call - Statement that directs control of the program to that function
\begin{verbatim}
     int main(){
          int sum = add(1, 5);
    	  cout<<"The sum is"<<sum<<"\n";
     }
\end{verbatim}

\subsection{Global vs Local Variables}
Global variables are declared outside all functions and may be accessed from anywhere whereas local variables are declared within its own block and may be accessed only within that block.\\


\end{document}