

\documentclass{article}

\usepackage{multicol}
\setlength{\columnsep}{1.2cm}
\usepackage{pgfplots}
\pgfplotsset{compat = newest}
\usepackage{titlesec}
\usepackage{graphicx}
\usepackage{wrapfig}
\usepackage{amsfonts}
\usepackage{tikz}
\usepackage{amssymb}
\usepackage{amsfonts}
\usepackage{amsmath}


\title{Exponential Functions}
\author{Anna Denisova}
\date{2023}

\begin{document}

\maketitle


%----------------------------------------------------------

\newpage
\section{Exponent Laws}
    
\begin{itemize}
    \begin{multicols}{3}
    \item $x^0 = 1$
    \item $x^1 = x$
    \item $x^ax^b = x^{a+b}$
    \item $\left(\frac{x}{y}\right)^a = \frac{x^a}{y^a}$
    \item $x^{-1} = \frac{1}{x}$
    \item $x^{-a} = \frac{1}{x^a}$
    \item $(x^a)^b = x^{ab}$
    \item $(ax^b)^c = a^cx^{bc}$
    \item $\left(\frac{x}{y}\right)^{-1} = \frac{y}{x}$
    \item $\left(\frac{x}{y}\right)^{-a} = \frac{y^a}{x^a}$
    \end{multicols}
\end{itemize}


%----------------------------------------------------------

%\newpage
\section{Radical Powerlaws}

\begin{itemize}
    \item $x^{\frac{1}{a}} = \sqrt[a]{x}$
    \item $x^{\frac{b}{a}} = \sqrt[a]{x^b}$
    \item $x^{-\frac{b}{a}} = \frac{1}{\sqrt[a]{x^b}}$
\end{itemize}

\subsection*{Steps to Solve:}
\begin{enumerate}
    \item Convert all radicals to fractional exponents
    \item Use powerlaws to simplify
    \item Convert back to root form
\end{enumerate}

%----------------------------------------------------------

%\newpage
\section{Rational Exponent Equations}
When solving rational exponenet equations, isolate the variable first then flip the exponent \\

In general:
\begin{align*}
    a &= b + cx^{\frac{e}{f}}\\
    \frac{a-b}{c} &= x^{\frac{e}{f}}\\
    \left(\frac{a-b}{c}\right)^{\frac{f}{e}} &= \left(x^{\frac{e}{f}}\right)^{\frac{f}{e}}\\
    \left(\frac{a-b}{c}\right)^{\frac{f}{e}} &= x
\end{align*}

%----------------------------------------------------------
%\newpage
\section{Exponential Equations}
When solving exponential equations:\\
\begin{enumerate}
    \item Get a single term on each side
    \item Make the bases match
    \item "Drop" the bases, set exponents equal
\end{enumerate}

\subsection*{Types can include:}
\begin{itemize}
    \item normal easy types
    \item divide first
    \item apples - $x^{a+b} = x^ax^b$
    \item hidden quadratic (trinomial) -  $x^{ab} = (x^a)^b$
\end{itemize}


%----------------------------------------------------------

%\newpage
\section{Properties of Exponential Functions}

\textbf{Exponential Function} - A function in the form $y=b^x$ where b can change your key points. The greater the b, the steeper. Any values make the function go through (0, 1) always. \\\\
\noindent
\textbf{Transformations} - $y = ab^{k(x-d)}+c$
%----------------------------------------------------------

%\newpage
%----------------------------------------------------------
%\newpage


%----------------------------------------------------------

\end{document}