\documentclass{article}

\usepackage{multicol}
\usepackage{titlesec}
\usepackage{graphicx}
\usepackage{wrapfig}

\graphicspath{{./images/}}

\title{RP2040 MACROPAD}
\author{Anna Denisova}
\date{March 2023}

\begin{document}

\maketitle
\tableofcontents

\newpage

\section{Introduction \& Objective}

The RP2040 Macropad is a raspberry pi microcontroller connected to twelve key pinouts, a small LED screen and one rotary encoder. In this project, we explore software made using the Arduino IDE and C++ code. 

\section{Theory}

\section{Abstract}

\section{Purpose}

The purpose of this project is to create software for the RP2040 Macropad.

\section{Hypothesis}

If I write specific code for the RP2040 Macropad then xyz will happen because of the software created.

\section{Variables}

Control
Dependent
Independent

\section{Materials}

\begin{itemize}
    \item Full kit with all parts found here: https://www.adafruit.com/product/5128 or alternatively:
    \item
        \begin{itemize}
            \item 3x4 Keys
            \item Encoder
            \item OLED display
            \item 12 Mechanical Key Switches
            \item 12 Keycaps
            \item Adafruit MacroPad bottom plate
            \item 3x4 Mechanical keyboard plate
            \item D-Shaft Skirted Rubber Knob
            \item 4 M3 5mm Machine screw
            \item 4 Rubber Feet
        \end{itemize}
    \item MacOS computer/laptop
    \item USB cord
\end{itemize}

\section{Procedure}

\begin{enumerate}
    \item Download the latest version of the Arduino IDE.
    \item Add the Philhower Board Manager URL to download additional boards including the RP2040 Macropad.
    \item In the Arduino IDE, go to Tools \> Board \> Boards Manager to select the RP2040 Macropad. 
    \item In the Sketch Tab, go to Port and select the corresponding port. 
    \item Now, the IDE is all set for code to be written in and run directly on the board. 
    \item The board itself can  be reset by pressing the reset button on the edge.
    \item To enter the bootloader, hold don on the rotary encoder and while continuiting to hold it, press and release the reset button. Continue holding the rotary encoder until the RPI-RP2 drive appears on your computer. 
    \item Now, you are all set to write code for the RP2040 Macropad. 
\end{enumerate}


\section{Experiment}

\section{Results \& Analysis}

\section{Conclusion \& Application}

\section{References}



\section{}

%-------------------------------------------------------------------------------------------------------------

\end{document}